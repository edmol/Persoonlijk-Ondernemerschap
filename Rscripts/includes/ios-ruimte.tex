\subsubsection{RUIMTE CRE\"EREN - Loslaten, veilig en vertrouwd voelen.}
Als je in staat bent om jezelf rust te geven, ontstaat er in een constructieve omgeving als vanzelf gevoelsruimte. Maar kijk uit, hier ligt je ego op de loer. Het zal je via je gevoel proberen te beperken, vast te houden en te overtuigen dat die ruimte alleen maar bedreigend is. Het ego zal elk gebaar van anderen in jouw richting negatief interpreteren door via destructieve gedachten\footnote{zie paragraaf 1.4 - kritieke faalfactoren} je gedrag en je houding te be\"invloeden. Hiermee houdt je ego zijn positie en stand. Want als anderen niet blij zijn met je gedrag, heeft je ego het gevaar voor veranderingen afgewimpeld.
\vspace{0.5 cm}
\newline
\includegraphics[width=\textwidth]{\Sexpr{img[105]}}
\newline
\includegraphics[width=\textwidth]{\Sexpr{img[41]}}
\newline
\includegraphics[width=\textwidth]{\Sexpr{img[42]}}
\newline
\includegraphics[width=\textwidth]{\Sexpr{img[43]}}
\newline
\includegraphics[width=\textwidth]{\Sexpr{img[44]}}
\newline
\includegraphics[width=\textwidth]{\Sexpr{img[45]}}
\newline
\includegraphics[width=\textwidth]{\Sexpr{img[46]}}
\newline
\includegraphics[width=\textwidth]{\Sexpr{img[47]}}
\newline
\includegraphics[width=\textwidth]{\Sexpr{img[48]}}
\newline
\includegraphics[width=\textwidth]{\Sexpr{img[49]}}
\newline
\includegraphics[width=\textwidth]{\Sexpr{img[50]}}
\newline
\includegraphics[width=\textwidth]{\Sexpr{img[123]}}
\newline
\includegraphics[width=\textwidth]{\Sexpr{img[124]}}
\newline
\includegraphics[width=\textwidth]{\Sexpr{img[125]}}