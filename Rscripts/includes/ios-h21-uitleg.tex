\subsection{Uitleg stellingen}
Jullie hebben onlangs samen de 100 stellingen van de team ontwikkel spiegel ingevuld. In dit rapport zijn jullie antwoorden verwerkt en gerubriceerd in verschillende categorie\"en. In de inhoudsopgave tref je de categorie\"en aan die in deze rapportage voor jullie zijn opgenomen. De rapportage is gebaseerd op het gedachtegoed dat staat verwoord in het boek 'In tien stappen op weg naar Persoonlijk Ondernemerschap' van Hans Heijnen. In hoofdstuk 4 van het boek wordt de achtergrond uitgebreider toegelicht. Je omgeving, je omstandigheden en je persoonlijke situatie veranderen voortdurend en daarmee ook jouw beeld van waar je nu staat. De rapportage kun je dan ook vergelijken met het nemen van een foto van jezelf, een momentopname, elke keer weer anders. De mate waarin je ondernemend en betrokken ben is dan ook geen kwestie van goed of fout, geschikt of ongeschikt.
\subsection{Uitleg antwoorden}
De antwoorden dragen bij aan het vergroten van de zelfkennis en zelfinzicht en geven inzicht in het groeipotentieel. Het ondersteunt je om op je werk zelf aan de slag te gaan met omstandigheden en mensen waar je mee van doen hebt. Je krijgt inzicht waar je het beste kunt beginnen om zo effectief een bijdrage te leveren aan de veranderingen of aan het cre\"eren van waarde voor jezelf en anderen. De samenvatting aan het einde geeft inzicht in waar het groeipotentieel het grootst is en dus ook wat het meest effectief is om mee aan de slag te gaan, zowel volgens jezelf als volgens het team.
\subsection{Uitleg verschillen}
De verschillen tussen hoe jullie zaken zien geven een aanvullend inzicht over het teambeeld, kennis en groeipotentieel. Met deze inzichten gaan jullie met elkaar in gesprek.\footnote{advies: Je veranderbegeleider kan jullie helpen bij het structureren van dit gesprek}. Een dialoog niet om het gelijk te halen maar om samen vast te stellen wat de beste en meest effectieve stappen zijn om het groeipotentieel om te zetten in daadwerkelijke capaciteit om betekenisvol te zijn. De samenvatting in het laatste hoofdstuk geven jullie per direct dit inzicht.
\subsection{Uitleg achtergrond en theorie}
De rapportage en grafieken en percentages geven aan waar jullie staan als het om jullie betrokkenheid versus ondernemendheid gaat ofwel: de mate waarin jullie zelf bewegen. Voorbeeld: is jouw percentage bij de categorie 'stappen' stap 1 : de regenboog zien laag bij betrokkenheid dan geeft dit aan dat je overwegend destructief of laag constructief hebt gescoord op de stellingen die hier betrekking op hebben ( oranje ).
Dit uit zich dan in een hoge groeinoodzaak. Praktisch wil dit zeggen dat je bij deze stap (te)weinig ondernemendheid laat zien en dat iets je weerhoud om te bewegen. De analyse per stelling geeft aan waar dit precies aan ligt. Hierbij geeft een volle balk per stelling een hoge mate van ondernemendheid of betrokkenheid aan en een lege balk het tegenovergestelde. De mate van {\bf vol} of {\bf leeg} zijn wordt bepaald door jouw scores per stelling : {\bf (a)ltijd, (m)eestal,(s)oms,(i)ncidenteel,(n)ooit}.
Neem rustig de tijd om de rapportage door te nemen en de uitkomsten op je in te laten inwerken. Afhankelijk van de aanleiding om deze rapportage te laten maken zet je nu je vervolgstappen. Als je vragen en of opmerkingen hebt neem dan gerust contact op.
\color{blue}
\subsection*{Aanvullende opmerkingen op de uitleg van deze rapportage}
\Sexpr{p24}
\color{black}